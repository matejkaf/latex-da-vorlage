\chapter{LaTeX Einführung}

Das vorliegende Dokument wurde in \LaTeX\ erstellt.
\LaTeX\ (gesprochen Latech) ist ein Textsatzsystem das speziell für umfangreiche  und komplexe wissenschaftliche, technische und mathematische Dokumente entwickelt wurde.
Man schreibt Quelltext wie bei einem Programm und übersetzt diesen Quelltext in ein PDF Dokument.
Siehe \cite{bib:latexintro}.
 
%~~~~~~~~~~~~~~~~~~~~~~~~~~~~~~~~~~~~~~~~~~~~~~~~~~~~~~~~~~~~~~~~~~~~~~~~~~~~~~~~
\section{Die Vorlage}

Die \LaTeX\ Quelltexte dieses Dokuments sind gedacht um als Vorlage für die eigenen Diplomarbeit verwendet zu werden.
Dazu muss der Inhalt durch die eigene Arbeit ersetzt werden.
\verb+Vorlage_DA.tex+ ist das zentrale Haupt-Dokument, in dieses werden die einzelnen Kapitel inkludiert. Die Dateien für die eingefügten Kapitel finden sich im Unterordner \verb+chapters+.

Die Vorlage kann von GitHub geladen werden:
\url{https://github.com/matejkaf/latex-da-vorlage}

%~~~~~~~~~~~~~~~~~~~~~~~~~~~~~~~~~~~~~~~~~~~~~~~~~~~~~~~~~~~~~~~~~~~~~~~~~~~~~~~~
\section{Programme}

Online (empfohlen):
Mit dem Service "`Overleaf"' (\url{https://www.overleaf.com/}). Damit können die Dokument online erstellt und gemeinsam bearbeitet werden.

Für Mac: MacTeX
\url{https://tug.org/mactex/}

Für Windows: MiKTeX
\url{http://miktex.org}\\
Zusätzlich auch die neueste Adobe Reader Version installieren!



%~~~~~~~~~~~~~~~~~~~~~~~~~~~~~~~~~~~~~~~~~~~~~~~~~~~~~~~~~~~~~~~~~~~~~~~~~~~~~~~~
\section{Bitmap Fonts}


Bei MiKTeX unter Windows kann es ein Font Problem geben.
Falls die Schrift nicht scharf ist --- PDF so vergrößern dass ein Buchstabe gut 10 cm groß ist --- dann sieht man die Pixel, siehe Abbildung \ref{fig:bitmapfont}.
In diesem Fall wird ein sogenannter Bitmap-Font verwendet. 
Besser ist ein Vektor-Font, dieser lässt sich beliebig ohne Qualitätsverlust vergrößern.

\textbf{Lösung:}
Mit "`MiKTex Package Manager"' das Package \verb+cm-super+ installieren.

\begin{figure}[H]
	\centering
	\includegraphics[width=0.75\textwidth]{./media/images/bitmap_font}
  	\caption{Oben Bitmap-, unten Vektor-Font}
  	\label{fig:bitmapfont}
\end{figure}

%~~~~~~~~~~~~~~~~~~~~~~~~~~~~~~~~~~~~~~~~~~~~~~~~~~~~~~~~~~~~~~~~~~~~~~~~~~~~~~~~
\section{LaTeX Quelltext}

Bei LaTeX wird der Text, dessen Gliederung in die Formatierung in puren Textfiles mit der Endung \verb+.tex+ beschrieben.
Die Zeichenkodierung der Files muss UTF-8 sein.

Der "`LaTeX Compiler"' (Programm mit dem Namen pdflatex) übersetzt diese Files in ein PDF Dokument.

%~~~~~~~~~~~~~~~~~~~~~~~~~~~~~~~~~~~~~~~~~~~~~~~~~~~~~~~~~~~~~~~~~~~~~~~~~~~~~~~~
\section{Gerüst}

Die Grundstruktur eines \LaTeX\ Dokuments:
\begin{Verbatim}[frame=single]
\documentclass[a4paper,10pt,final,oneside]{scrartcl}	

\usepackage{anysize}
\usepackage[utf8]{inputenc}
\usepackage[T1]{fontenc}
\usepackage[ngerman]{babel}
\usepackage[pdftex]{hyperref}
\usepackage{url}
 
\begin{document}

Ab hier kommt der Inhalt

\end{document}
\end{Verbatim}

Hinweis:
Durch die Vorlage ist diese Grund-Struktur bereits vorgegeben.

%~~~~~~~~~~~~~~~~~~~~~~~~~~~~~~~~~~~~~~~~~~~~~~~~~~~~~~~~~~~~~~~~~~~~~~~~~~~~~~~~
\section{Formatierungen}

\begin{minipage}[t]{0.6\linewidth}
\begin{Verbatim}[frame=single]
Mehrere     Leerzeichen werden 
ignoriert
ein einfacher Zeilenumbruch gilt 
als Leerzeichen.

Eine leere Zeile kennzeichnet einen 
neuen Absatz.
\end{Verbatim}
\end{minipage}
\begin{minipage}[t]{0.4\linewidth}
Mehrere     Leerzeichen werden ignoriert
ein einfacher Zeilenumbruch 
gilt als Leerzeichen.

Eine leere Zeile kennzeichnet 
einen neuen Absatz.
\end{minipage}

\vspace{1em}\noindent
Diverse Formatierungen:

\begin{tabular}{l|l}
\texttt{Typewriter} & \lstinline!\texttt{...}! o.\ \lstinline!{\ttfamily ...}!\\ 
\textbf{Fett} & \lstinline!\textbf{...}! o.\ \lstinline!{\bfseries ...}!\\ 
\textit{Italic} & \lstinline!\textit{...}! o.\ \lstinline!{\itshape ...}!\\ 
\textsl{Slanted} & \lstinline!\textsl{...}! o.\ \lstinline!{\slshape ...}!\\ 
\textsc{Kapitälchen} & \lstinline!\textsc{...}! o.\ \lstinline!{\scshape ...}!\\ 
\textmd{Normal} & \lstinline!\textmd{...}! o.\ \lstinline!{\mdseries ...}!\\ 
\emph{Emph} & \lstinline!\emph{...}! o.\ \lstinline!{\em ...}!\\ 
\textsf{Sans Serif} & \lstinline!\textsf{...}! o.\ \lstinline!{\sffamily ...}!\\
\underline{Unterstrichen} & \lstinline!\underline{...}!\\
Größen & \lstinline!\tiny! \lstinline!\scriptsize! \lstinline!\footnotesize! \lstinline!\small! \lstinline!\normalsize! \lstinline!\large! \lstinline!\Large!\\
 & \lstinline!\LARGE! \lstinline!\huge! \lstinline!\Huge!\\ 
Zentriert & \lstinline!\begin{center}...\end{center}!\\
Gesch.\ Leerz. & \lstinline!~!\\
Zeilenumbruch & \lstinline!\\! oder \lstinline!\newline!\\
Absatzumbruch & \lstinline!\par! oder leere Zeile.\\

\end{tabular}


%~~~~~~~~~~~~~~~~~~~~~~~~~~~~~~~~~~~~~~~~~~~~~~~~~~~~~~~~~~~~~~~~~~~~~~~~~~~~~~~~
\section{Überschriften}

\begin{Verbatim}[frame=single]
\chapter{Hauptkapitel}
\section{Kapitel}
\subsection{Unterkapitel}
\subsubsection{Unterunterkapitel}
\end{Verbatim}

Die Kapitelnummerierung und das Inhaltsverzeichnis werden automatisch erstellt.

%~~~~~~~~~~~~~~~~~~~~~~~~~~~~~~~~~~~~~~~~~~~~~~~~~~~~~~~~~~~~~~~~~~~~~~~~~~~~~~~~
\section{Autoren}

Für eine Diplomarbeit mit mehreren Autoren muss nachvollziehbar sein wer für welchen Teil der Urheber ist.
Dies ist durch \textbf{Namenskürzel} (2-stellig) in den Überschriften darzustellen (siehe in diesem Dokument).

Eine Person soll immer für ein komplettes Hauptkapitel (\verb+chapter+) oder Kapitel (\verb+section+) verantwortlich sein. 
Eine Aufteilung auf Ebene der Unterkapitel (\verb+subsection+) oder Unterunterkapitel (\verb+subsubsection+) sollte vermieden werden.

Die Namenskürzel werden in \verb+Vorlage_DA.tex+ definiert:

\begin{Verbatim}[frame=single]
% Initialen der Authoren
\def\authorInitialsA{MM} % Max Mustermann
\def\authorInitialsB{FF} % Frieda Fröhlich
\def\authorInitialsC{FE} % Fritz Einstein
\def\authorInitialsD{WA} % Weiterer Autor
\end{Verbatim}
 

Für das Einfügen der Namenskürzel ins Dokument dienen in weiterer Folge die Befehle 
\verb+\authorA+, 
\verb+\authorB+, 
\verb+\authorC+, 
\verb+\authorD+.

Beispiel --- erzeugt die Überschrift dieses Kapitels:
\begin{Verbatim}[frame=single]
\section{Autoren\authorB}
\end{Verbatim}

%~~~~~~~~~~~~~~~~~~~~~~~~~~~~~~~~~~~~~~~~~~~~~~~~~~~~~~~~~~~~~~~~~~~~~~~~~~~~~~~~
\section{Bilder einfügen}

Formate: pdf, jpg und png.
Dateien im Verzeichnis \verb+media/images+ ablegen.

\begin{Verbatim}[frame=single]
In Abbildung \ref{fig:htl01} sieht man das Logo der HTL Braunau.
\begin{figure}[H]
  \centering
  \includegraphics[width=0.3\textwidth]{./media/images/htl_c_cmyk_rein.pdf}
  \caption{Logo der HTL Braunau.}
  \label{fig:htl01}
\end{figure}
\end{Verbatim}

\begin{framed}
In Abbildung \ref{fig:htl01} sieht man das Logo der HTL Braunau.
\begin{figure}[H]
	\centering
	\includegraphics[width=0.3\textwidth]{./media/images/htl_c_cmyk_rein.pdf}
  	\caption{Logo der HTL Braunau.}
  	\label{fig:htl01}
\end{figure}
\end{framed}

Hinweis: Dateipfade mit \verb+"/"+ bilden!

Siehe
\url{http://en.wikibooks.org/wiki/LaTeX/Importing_Graphics}

Der Befehl \lstinline{\label} gibt der Abbildung einen eindeutigen Namen.
Durch \lstinline{\ref} wird dieser Name referenziert, d.h. es wird die automatisch generierte Abbildungsnummer eingefügt (dazu muss das \LaTeX\ Dokument 2-mal erstellt werden!)

Siehe \ref{ref:abbildungen}, Seite \pageref{ref:abbildungen} für Abbildungen die mehrere Bilder enthalten.

%~~~~~~~~~~~~~~~~~~~~~~~~~~~~~~~~~~~~~~~~~~~~~~~~~~~~~~~~~~~~~~~~~~~~~~~~~~~~~~~~
\section{Querverweise}

Mit Hilfe von Querverweisen verweist man auf andere Stellen im Dokument. 
Z.B. in der Form: "`Siehe \ref{ref:abbildungen}"' --- es wird die Kapitelnummer bzw. Abbildungsnummer angegeben.

Es kann auf Abbildungen und auf Überschriften verwiesen werden.
Diese erhalten zuerst mit \lstinline{\label} einen Namen.
\begin{Verbatim}[frame=single]
\section{Kapitelname} \label{ref:meinebezeichnung}
\end{Verbatim}

Möchte man auf diese Elemente verweisen gibt man den Namen im \lstinline{\ref} Befehl an.
\begin{Verbatim}[frame=single]
Eine genaue Beschreibung dieses Themas ist 
in Kapitel \ref{ref:meinebezeichnung} zu finden.
\end{Verbatim}

Ein Doppelpunkt als Teil des Namens ist erlaubt. Auf diese Weise können zum Beispiel Namen von Abbildungen, Überschriften und Listings unterschieden werden (\mbox{\lstinline{fig:}/\lstinline{ref:}/\lstinline{code:}}).

\LaTeX\ macht aus Querverweisen automatisch PDF Links.


%~~~~~~~~~~~~~~~~~~~~~~~~~~~~~~~~~~~~~~~~~~~~~~~~~~~~~~~~~~~~~~~~~~~~~~~~~~~~~~~~
\section{Aufzählungen}

\begin{minipage}[c]{0.3\linewidth}
\begin{Verbatim}[frame=single]
\begin{itemize}
\item Eins
\item Zwei
\item Drei
\end{itemize}
\end{Verbatim}
\end{minipage}
\begin{minipage}[c]{0.5\linewidth}
\begin{itemize}
\item Eins
\item Zwei
\item Drei
\end{itemize}
\end{minipage}

\noindent
\begin{minipage}[c]{0.3\linewidth}
\begin{Verbatim}[frame=single]
\begin{enumerate}
\item Eins
\item Zwei
\item Drei
\end{enumerate}
\end{Verbatim}
\end{minipage}
\begin{minipage}[c]{0.5\linewidth}
\begin{enumerate}
\item Eins
\item Zwei
\item Drei
\end{enumerate}
\end{minipage}

%~~~~~~~~~~~~~~~~~~~~~~~~~~~~~~~~~~~~~~~~~~~~~~~~~~~~~~~~~~~~~~~~~~~~~~~~~~~~~~~~
\section{Mathematische Formeln}

\begin{Verbatim}[frame=single]
Abgesetzte Formel:
\begin{equation*}
\frac{1+x}{1-x} \cdot \sqrt[3]{2} \cdot \binom{n}{k}
\end{equation*}
\end{Verbatim}

Abgesetzte Formel:
\begin{equation*}
\frac{1+x}{1-x} \cdot \sqrt[3]{2} \cdot \binom{n}{k}
\end{equation*}

\begin{Verbatim}[frame=single]
Formel im Textfluss:
$\frac{1+x}{1-x}, \sqrt[3]{2}, \binom{n}{k}$
\end{Verbatim}

Formel im Textfluß: $\frac{1+x}{1-x}, \sqrt[3]{2}, \binom{n}{k}$
\\

%~~~~~~~~~~~~~~~~~~~~~~~~~~~~~~~~~~~~~~~~~~~~~~~~~~~~~~~~~~~~~~~~~~~~~~~~~~~~~~~~
\section{Programm Quelltext}

Die Umgebung lstlisting übernimmt das Formatieren von Programmquelltext.

\begin{minipage}[t]{\linewidth}
\begin{Verbatim}[frame=single]
Listing \ref{code:complex} zeigt die Implementierung eines 
besonders komplexen Algorithmus.
\begin{lstlisting}[
    language=java,
    caption={Komplizierter Quelltext.},
    label=code:complex
]
while(x>0) {
	x--;
	// bla bla
}
\end{lstlisting}
\end{Verbatim}
\end{minipage}

\begin{framed}
Listing \ref{code:complex} zeigt die Implementierung eines 
besonders komplexen Algorithmus.
\begin{lstlisting}[
    language=java,
    caption={Komplizierter Quelltext.},
    label=code:complex
]
while(x>0) {
	x--;
	// bla bla
}
\end{lstlisting}
\end{framed}

Siehe auch \url{https://en.wikibooks.org/wiki/LaTeX/Source_Code_Listings}

%~~~~~~~~~~~~~~~~~~~~~~~~~~~~~~~~~~~~~~~~~~~~~~~~~~~~~~~~~~~~~~~~~~~~~~~~~~~~~~~~
\section{Code im Text}

\begin{minipage}[t]{\linewidth}
\begin{Verbatim}[frame=single]
Mit dem Befehl lstinline können kurze Programmfragmente 
direkt in den Textfluss integriert werden.
Das sieht dann so aus: 
\lstinline[language=java]{labels.add("" + i)}.
Allzu lange sollten diese Programmausschnitte aber nicht sein.
\end{Verbatim}
\end{minipage}

Mit dem Befehl lstinline können kurze Programmfragmente 
direkt in den Textfluß integriert werden.
Das sieht dann so aus: 
\lstinline{labels.add("" + i)}.
Allzu lange sollten diese Programmausschnitte aber nicht sein.


%~~~~~~~~~~~~~~~~~~~~~~~~~~~~~~~~~~~~~~~~~~~~~~~~~~~~~~~~~~~~~~~~~~~~~~~~~~~~~~~~
\section{Seitenümbruche}

Ein Seitenumbruch kann mit \lstinline{\pagebreak} erzwungen werden.

Soll etwas als ganzes auf der Seite stehen und nicht umgebrochen werden, so kann dieser Teil in eine \lstinline{minipage} Umgebung eingeschlossen werden.

\begin{Verbatim}[frame=single]
\begin{minipage}{\linewidth}
In diesem Teil findet kein Seitenumbruch statt.
\end{minipage}
\end{Verbatim}

Häufig wird dies bei Programmlistings nötig sein:

\begin{Verbatim}[frame=single]
Das Listing \ref{code:codenopagebreak} ist in eine minipage 
eingebunden. Dies wirkt sich so aus, dass dieses Listing nur 
als ganzes auf der Seite steht, sollte es nicht mehr Platz 
haben wird es komplett auf die folgende Seite gesetzt und 
es entsteht ein Leerraum auf der vorhergehenden Seite.

\begin{minipage}{\linewidth}
\begin{lstlisting}[
	language=java,
	caption={Java Quelltext.},
	label=code:codeexample1
]
public void prepend(Node n) {
	n.next=start;
	start=n;
}
\end{lstlisting}
\end{minipage}
\end{Verbatim}

Das Listing \ref{code:codenopagebreak} ist in eine minipage eingebunden.
Dies wirkt sich so aus, dass dieses Listing nur als ganzes auf der Seite steht, sollte es nicht mehr Platz haben wird es komplett auf die folgende Seite gesetzt und es entsteht ein Leerraum auf der vorhergehenden Seite.

\begin{minipage}{\linewidth}
\begin{lstlisting}[
	language=java,
	caption={Java Quelltext.},
	label=code:codenopagebreak
]
public void prepend(Node n) {
	n.next=start;
	start=n;
}
\end{lstlisting}
\end{minipage}


%~~~~~~~~~~~~~~~~~~~~~~~~~~~~~~~~~~~~~~~~~~~~~~~~~~~~~~~~~~~~~~~~~~~~~~~~~~~~~~~~
\section{Quellen und Literatur}

Alle Quellen befinden sich im Dokument \verb+literature.bib+, in einem sogennanten BibTeX Format.
Jede Quelle erhält einen Namen, z.B. \lstinline|bib:latexintro|.

Im Text wird durch den Befehl \lstinline{\cite} zitiert.
Bsp.:

\begin{Verbatim}[frame=single]
Siehe \cite{bib:latexintro}.
\end{Verbatim}

\begin{framed}
Siehe \cite{bib:latexintro}.
\end{framed}



%~~~~~~~~~~~~~~~~~~~~~~~~~~~~~~~~~~~~~~~~~~~~~~~~~~~~~~~~~~~~~~~~~~~~~~~~~~~~~~~~
\section{Links}

Links können in der Form einer URL eingefügt werden.
Dies kann (in kleinerem Umfang) statt Einträgen im Literaturverzeichnis verwendet werden.

\begin{Verbatim}[frame=single]
\url{http://www.orf.at}
\end{Verbatim}
\begin{framed}
\url{http://www.orf.at}
\end{framed}

Lange URL's sehen nicht besonders gut aus. Etwa \url{https://www.youtube.com/watch?v=_vQaOvPsLko&list=PL6gx4Cwl9DGBsvRxJJOzG4r4k_zLKrnxl&index=49}.

In diesem Fall bietet es sich an solche URL's mit einem Kurz-URL-Dienst (z.B. \url{tinyurl.com}) zu verkürzen. 
Obige URL in verkürzter Form: \url{http://tinyurl.com/hqtzqan}


%~~~~~~~~~~~~~~~~~~~~~~~~~~~~~~~~~~~~~~~~~~~~~~~~~~~~~~~~~~~~~~~~~~~~~~~~~~~~~~~~
\section{Fußnoten}

\begin{Verbatim}[frame=single]
Eine Fußnote kann man einfach%
\footnote{Hier zum Beispiel}
irgendwo in den Text einfuegen.
\end{Verbatim}

Eine Fußnote kann man einfach%
\footnote{Hier zum Beispiel}
irgendwo in den Text einfügen, diese wird an das untere Ende der Seite gesetzt.


%~~~~~~~~~~~~~~~~~~~~~~~~~~~~~~~~~~~~~~~~~~~~~~~~~~~~~~~~~~~~~~~~~~~~~~~~~~~~~~~~
\section{Tabellen}

Mit der Umgebung \lstinline{tabular} bzw. erweitert: \lstinline{tabularx}.

Siehe \url{https://en.wikibooks.org/wiki/LaTeX/Tables}

%: TODO Tabellen

%~~~~~~~~~~~~~~~~~~~~~~~~~~~~~~~~~~~~~~~~~~~~~~~~~~~~~~~~~~~~~~~~~~~~~~~~~~~~~~~~
\section{Mehrspaltiger Text}

\begin{multicols}{2}

Um einen zweispaltigen Text zu erzeugen kann die \verb+multicols+ Umgebung verwendet werden

\begin{Verbatim}[frame=single]
\begin{multicols}{2}
Text in 2 Spalten
\end{multicols}
\end{Verbatim}

Hier wurde multicols verwendet um die Abbildung platzsparend neben dem Text zu platzieren.
Siehe Logo der HTL Braunau in Abbildung \ref{logoTwoCols}.

%	--------------
%	Eine Abbildung

\columnbreak
\begin{figure}[H]
	\centering
	\includegraphics[width=0.3\textwidth]{./media/images/htl_c_cmyk_rein.pdf}
  	\caption{Logo auf der rechten Seite.}
  	\label{logoTwoCols}
\end{figure}

Um einen Spaltenumbruch an einer bestimmten Stelle zu erzwingen:
\begin{Verbatim}[frame=single]
\columnbreak
\end{Verbatim}


\end{multicols}


